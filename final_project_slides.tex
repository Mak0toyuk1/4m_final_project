\documentclass{beamer}
\usepackage[authoryear]{natbib}
\usepackage{graphicx}
%Information to be included in the title page:
\title{Multivariate analysis of the diabetes dataset}
\author{Tony Xu, Safi Khan, Rayyan Kazim, Xinyi Chen, Zesen Chen}
\institute{McMaster University}
\date{15/11/2024}

\begin{document}

\frame{\titlepage}

\begin{frame}
\frametitle{Dataset}
\begin{itemize}
    \setlength\itemsep{3em}
    \item We will study the dataset diabetes obtained from kaggles \cite{Kaggles}
    \item The dataset contains $768$ rows and $9$ columns.
    \item Each row corresponds to an unique patient record.
    \item We will be using the R programming language.
\end{itemize}
\end{frame}

\begin{frame}
    \frametitle{Exploratory Data Analysis (EDA) and data preparation}
    %insert all our tables
    \begin{itemize}
        \setlength\itemsep{1em}
        \item All variables are integers except for "BMI" and "DiabetesPedigreeFunction"
        \item "SkinThickness" is well correlated with "BMI" and "Insulin", "Age" is well correlated with "Pregnancies".
        \item "Glucose" is reasonably correlated with "insulin", "BMI" and "Age".
        \item The dataset will be split into 2, with the response variable being "outcome" and all other variables being predictor variables.
        \item 75 percent of the data will be used for training, and the rest will be used for testing.
    \end{itemize}
\end{frame}

\begin{frame}
    \frametitle{Methodologies - Supervised learning analysis}
        \begin{itemize}
            \setlength\itemsep{1em}
            \item We used the methods: k-nearest neighbours \cite{peterson2009k},random forest classfiers \cite{zhou2012ensemble} and boosting \cite{chen2015xgboost} for our supervised learning analysis.
            \item K-nearest neighbours \cite{peterson2009k} is a non-parametric, supervised learning classifiers that uses proximity to make classifications about the grouping of a dataset.
            \item RandomForest classifiers \cite{zhou2012ensemble} is a bootstrapping sampling method that combines the results of multiple decision trees to draw on a conclusion.
            \item Boosting \cite{chen2015xgboost} is similar to random forest, however it is not a bootstrapping sampling method. Boosting also uses the entire dataset, or some subsample thereof, to generate the ensemble.
        \end{itemize}
\end{frame}

\begin{frame}
    \frametitle{Methodologies - Logistic regression}
        \begin{itemize}
            \setlength\itemsep{1em}
            \item We will perform a binary logistic regression \cite{faraway2016extending} since our response variable "outcome" is binary.
            \item The initial model incorporates all eight predictor variables.
            \item The objective of this technique is to identify the most significant predictor using backwards elimination.
            \item We want the final model to have only the most significant variables.
        \end{itemize}
\end{frame}

\begin{frame}
    \frametitle{Discussions - Logistic regression}
        \begin{itemize}
            \setlength\itemsep{1em}
            \item Removed variables "SkinThickness" and "insulin" from the model, since their p-values were greater than $0.05$.
            \item We have evidence to suggest that "Pregnancies", "Glucose", "Blood Pressure", "BMI", "DiabetesPedigreeFunction" and "Age" have a significant influence over "outcome".
        \end{itemize}
\end{frame}

\begin{frame}
    \frametitle{Discussions - k-nearest neighbours}
    %insert the plot here.
        \begin{itemize}
            \setlength\itemsep{1em}
            \item 5-fold cross validation suggests that the best value for $k$ is $k=3$
            \item Executing the knn() function with $k=3$, we obtain that the MCR of the k-nearest neighbours is $0.2916667$
        \end{itemize}
\end{frame}

\begin{frame}
    \frametitle{Discussions - Random forests}
    %insert the variable importance plot here.
        \begin{itemize}
            \setlength\itemsep{1em}
            \item 5-fold cross validation suggests that the best value for mtry is $4$ and the best value for ntree is $200$.
            \item Executing the RandomForest() function with mtry $=4$ and ntree $=200$, we obtain that the MCR of the RandomForest is $0.28125$.
            \item We observe that Glucose and BMI are the two most important variables.
        \end{itemize}
\end{frame}

\begin{frame}
    \frametitle{Conclusion}
    %insert the variable importance plot here.
        This section is under construction.
\end{frame}

\begin{frame}
    \frametitle{References}
    \bibliographystyle{apa} 
    \bibliography{references}
\end{frame}

\end{document}
